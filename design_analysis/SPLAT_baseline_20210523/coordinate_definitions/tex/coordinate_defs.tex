\documentclass[]{article}
\usepackage{booktabs}
\usepackage{hyperref}

\title{SPLAT Baseline Coordinate Definitions}
\author{P. Gallardo}
\date{\today}

\begin{document}
\maketitle

In this document I summarize the coordinate
definitions for the three mirrors in the TMA design 
filename  SPLAT\textunderscore Base \textunderscore Fwd.zmx, which can be found in \url{https://github.com/patogallardo/zemax_tools/tree/master/design_analysis/SPLAT_baseline_20210523/
SPLAT_Base_Fwd.zmx}. For Step files of this design see folder coordinate\textunderscore definitions/step\textunderscore files. 

Surfaces covered in this document are: Origin, Primary, Secondary, Tertiary and Image surface. These surfaces can be fully defined with a location vector and a rotation angle. Another way of defining these is by using a location vector and a rotation matrix, which is useful for Grasp studies.

\section{Surface definitions}

Table \ref{surfdeftable} shows the surface definitions for this design. Angle $\alpha$ refers to the rotation around the X coordinate in degrees.

\begin{center}
\begin{table}
\begin{tabular}{l c c c c}
Surface       &  X[mm]           &   Y[mm]           &   Z[mm]         &  $\alpha$[deg]   \\
Origin        &  0               &   0               &   0             &  0               \\
Prime (M1)    &  0               &   0               &   0             &  1.554018105E+02 \\
Secondary (M2)&  0               &   5.615000000E+03 & 4.898000000E+03 &  1.711645903E+02 \\ 
Tertiary (M3) &  0               &  8.367000000E+03  & 4.450000000E+02 & -1.708825424E+02 \\
Image         &  0               &   9.575761967E+03 & 5.024441463E+03 &  1.689266466E+02 \\                 
\end{tabular}
\label{surfdeftable}
\caption{Surface definition coordinates for the TMA.}
\end{table}
\end{center}

\begin{table}
\begin{center}
	\begin{tabular}{lrrrr}
\toprule
            surface &  X[mm] &    Y[mm] &    Z[mm] &  $\alpha$ [deg] \\
\midrule
              prime &  0.000 &    0.000 &    0.000 &      155.402 \\
             second &  0.000 & 5615.000 & 4898.000 &      171.165 \\
               tert &  0.000 & 8367.000 &  445.000 &     -170.883 \\
Front of cryo plate &  0.000 & 9570.000 & 4995.000 &      168.927 \\
\bottomrule
\end{tabular}

\end{center}
\end{table}

\begin{table}
\begin{center}
	\begin{tabular}{rrr}
\toprule
1.00000 &  0.00000 &  0.00000 \\
0.00000 & -0.90925 & -0.41625 \\
0.00000 &  0.41625 & -0.90925 \\
\bottomrule
\end{tabular}

\end{center}
\end{table}


\begin{table}
\begin{center}
	\begin{tabular}{rrr}
\toprule
1.00000 &  0.00000 &  0.00000 \\
0.00000 & -0.98813 & -0.15360 \\
0.00000 &  0.15360 & -0.98813 \\
\bottomrule
\end{tabular}

\end{center}
\end{table}


\begin{table}
\begin{center}
	\begin{tabular}{rrr}
\toprule
1.00000 &  0.00000 &  0.00000 \\
0.00000 & -0.98737 &  0.15846 \\
0.00000 & -0.15846 & -0.98737 \\
\bottomrule
\end{tabular}

\end{center}
\end{table}


\begin{table}
\begin{center}
	\input{Image_rotmat.tex}
\end{center}
\end{table}

\end{document}