\documentclass[12pt, letterpaper, twoside]{article}
\usepackage[utf8]{inputenc}
\usepackage{booktabs}
\usepackage{graphicx}

\title{TMP mirror definitions}
\author{Patricio Gallardo}
%\date{February 2014}

\begin{document}


\maketitle


\section{Mirror Sag}

Mirror Sag is implemented in Zemax as extended polynomials (from a flat surface) of 14 or 20 terms. From these 20 terms most of them are zero. The sag of the mirror is defined following 
\begin{equation}
\label{eq:1}
z(x,y) = \sum_{i=0}^5 \sum_{j=0}^{5} p_{i,j} \left( \frac{x}{R_{max}}\right )^i \left(\frac{y}{R_{max}}\right)^j [\rm{mm}]. 
\end{equation}

Here the sum covers 36 elements which define a general order 5 polynomial. This polynomial expansion can be evaluated using the terms shown in tables \ref{tab_prime}, \ref{tab_second} and \ref{tab_tert}. The term $R_{max}$ is a normalization length-scale which equals 2500mm.

\begin{table}
	\begin{tabular}{lrrrrrr}
\toprule
{} &         j=0 &       j=1 &         j=2 &       j=3 &       j=4 &  j=5 \\
\midrule
i=0 &    0.000000 & -4.805417 & -114.281586 &  3.971894 & -0.096818 &  0.0 \\
i=1 &    0.000000 &  0.000000 &    0.000000 &  0.000000 &  0.000000 &  0.0 \\
i=2 & -138.667041 &  5.410397 &    0.118027 &  0.000000 &  0.000000 &  0.0 \\
i=3 &    0.000000 &  0.000000 &    0.000000 &  0.000000 &  0.000000 &  0.0 \\
i=4 &    0.251763 &  0.000000 &    0.000000 &  0.000000 &  0.000000 &  0.0 \\
i=5 &    0.000000 &  0.000000 &    0.000000 &  0.000000 &  0.000000 &  0.0 \\
\bottomrule
\end{tabular}

	\caption{Primary mirror definition according to \ref{eq:1}. Mirror rim has a semiwidth of 2520mm in the x direction and 2850mm in the y direction with a decenter of 65mm in the y direction.}
		\label{tab_prime}
\end{table}

\begin{table}

	\begin{tabular}{lrrrrrr}
\toprule
{} &         j=0 &        j=1 &        j=2 &        j=3 &       j=4 &  j=5 \\
\midrule
i=0 &    0.000000 &   0.000000 & -142.89385 &  16.542098 & -1.748626 &  0.0 \\
i=1 &    0.000000 &   0.000000 &    0.00000 &   0.000000 &  0.000000 &  0.0 \\
i=2 & -302.813618 &  45.268179 &   -0.90499 &   0.000000 &  0.000000 &  0.0 \\
i=3 &    0.000000 &   0.000000 &    0.00000 &   0.000000 &  0.000000 &  0.0 \\
i=4 &    8.222494 &   0.000000 &    0.00000 &   0.000000 &  0.000000 &  0.0 \\
i=5 &    0.000000 &   0.000000 &    0.00000 &   0.000000 &  0.000000 &  0.0 \\
\bottomrule
\end{tabular}

	\caption{Secondary mirror definition according to \ref{eq:1}. Mirror rim has a semiwidth of 1780mm in the x direction and 2430mm in the y direction with a decenter of 150mm in the y direction.}
	\label{tab_second}
\end{table}

\begin{table}
	\documentclass[convert={convertexe={magick.exe}}]{standalone}
\usepackage{cmbright}

\begin{document}

$ z(x, y) = C_{X0Y1} \, y + 
C_{X2Y0} \, x^2 + 
C_{X0Y2} \, y^2 + 
C_{X2Y1} \, x^2y^1+ 
C_{X0Y3} \, y^3 + 
C_{X4Y0} \, x^4 + 
C_{X2Y2} \, x^2y^2 + 
C_{X0Y4} \, y^4 + 
C_{X4Y1} \, x^4y + 
C_{X2Y3} \, x^2y^3 + 
C_{X0Y5} \, y^5  $

\end{document}
	\caption{Tertiary mirror definition according to \ref{eq:1}. Mirror rim has a semiwidth of 2690 mm in the x direction and 2780 mm in the y direction with a decenter of 110 mm in the y direction.}
	\label{tab_tert}
\end{table}


\begin{figure}
	\centering
	\includegraphics[width=0.5\textwidth]{./prime.png}\hfill
	\includegraphics[width=0.5\textwidth]{./second.png}
	
	\includegraphics[width=0.5\textwidth]{./tert.png}
	\label{fig:mirrorshape}
	\caption{Mirror sag defined with matrix elements accodring to tables 1-3.}
\end{figure}

\begin{figure}
	\centering
	\includegraphics[width=0.9\textwidth]{mirror_fp_rays.PNG}
	\caption{TMP ray trace}
\end{figure}

\end{document}