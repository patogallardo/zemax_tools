\documentclass{article}
\usepackage[a4paper,margin=0.5in,landscape]{geometry}
\usepackage{booktabs}

\setlength{\tabcolsep}{2pt}
\renewcommand{\arraystretch}{2}

\begin{document}

\newcommand{\zernikeeq}{\begin{equation}
	z_{zernike} = \frac{c_x x^2 + c_y y^2}{1+ \sqrt{1-(1-k_x)c_x^2 x^2-(1+k_y)c_y^2y^2}} + \sum_{i=1}^{16} \alpha_i x^i + \sum_{j=1}^{16} \beta_j y^j \qquad \rm{,with\ }  c_x=\frac{1}{R_x} \qquad{and\ } c_y=\frac{1}{R_y}
\end{equation}
}


\section*{Lens 1}

\begin{table}[h]
\tiny
	\begin{tabular}{lrrrrrrr}
\toprule
       SurfType &     Radius &  Thickness &  Semi-Diameter &       Conic &   X Radius &     X Conic &  Norm Radius \\
\midrule
Biconic Zernike & 9.5485e+02 & 1.0206e+01 &     1.0000e+02 & -1.2443e+00 & 8.3774e+02 & -4.4373e+01 &   1.0000e+02 \\
\bottomrule
\end{tabular}

\end{table}
\begin{table}[h]
	\tiny
	\begin{tabular}{rrrrrrrrrrrrrrrr}
\toprule
       X\textasciicircum 1 &        X\textasciicircum 2 &        X\textasciicircum 3 &        X\textasciicircum 4 &        X\textasciicircum 5 &         X\textasciicircum 6 &        X\textasciicircum 7 &         X\textasciicircum 8 &        X\textasciicircum 9 &        X\textasciicircum 10 &       X\textasciicircum 11 &        X\textasciicircum 12 &       X\textasciicircum 13 &       X\textasciicircum 14 &       X\textasciicircum 15 &       X\textasciicircum 16 \\
\midrule
0.0000e+00 & 4.7124e-06 & 0.0000e+00 & 9.4851e-11 & 0.0000e+00 & -4.2154e-15 & 0.0000e+00 & -1.4746e-18 & 0.0000e+00 & -3.0640e-22 & 0.0000e+00 & -6.5879e-28 & 0.0000e+00 & 0.0000e+00 & 0.0000e+00 & 0.0000e+00 \\
\bottomrule
\end{tabular}

\end{table}
\begin{table}[h]
	\tiny
	\begin{tabular}{rrrrrrrrrrrrrrrr}
\toprule
       Y\textasciicircum 1 &        Y\textasciicircum 2 &        Y\textasciicircum 3 &        Y\textasciicircum 4 &        Y\textasciicircum 5 &         Y\textasciicircum 6 &        Y\textasciicircum 7 &         Y\textasciicircum 8 &        Y\textasciicircum 9 &        Y\textasciicircum 10 &       Y\textasciicircum 11 &       Y\textasciicircum 12 &       Y\textasciicircum 13 &       Y\textasciicircum 14 &       Y\textasciicircum 15 &       Y\textasciicircum 16 \\
\midrule
0.0000e+00 & 6.7277e-12 & 0.0000e+00 & 4.4299e-14 & 0.0000e+00 & -9.3278e-15 & 0.0000e+00 & -9.1121e-19 & 0.0000e+00 & -9.1122e-23 & 0.0000e+00 & 2.1065e-28 & 0.0000e+00 & 0.0000e+00 & 0.0000e+00 & 0.0000e+00 \\
\bottomrule
\end{tabular}

\end{table}


Biconic Zernike polynomial

\zernikeeq

\newpage
\section*{Lens 2}

\begin{table}[h]
	\tiny
	\begin{tabular}{lrrrrrrr}
\toprule
       SurfType &      Radius &  Thickness &  Semi-Diameter &       Conic &    X Radius &     X Conic &  Norm Radius \\
\midrule
Biconic Zernike & -4.2193e+02 & 0.0000e+00 &     8.5113e+01 & -6.2760e+00 & -4.3229e+02 & -5.3555e+00 &   1.0000e+02 \\
\bottomrule
\end{tabular}

\end{table}
\begin{table}[h]
	\tiny
	\begin{tabular}{rrrrrrrrrrrrrrrr}
\toprule
       X\textasciicircum 1 &         X\textasciicircum 2 &        X\textasciicircum 3 &        X\textasciicircum 4 &        X\textasciicircum 5 &         X\textasciicircum 6 &        X\textasciicircum 7 &        X\textasciicircum 8 &        X\textasciicircum 9 &       X\textasciicircum 10 &       X\textasciicircum 11 &        X\textasciicircum 12 &       X\textasciicircum 13 &       X\textasciicircum 14 &       X\textasciicircum 15 &       X\textasciicircum 16 \\
\midrule
0.0000e+00 & -7.0079e-08 & 0.0000e+00 & 3.8447e-11 & 0.0000e+00 & -1.6131e-13 & 0.0000e+00 & 5.6335e-21 & 0.0000e+00 & 4.4621e-25 & 0.0000e+00 & -3.9332e-26 & 0.0000e+00 & 0.0000e+00 & 0.0000e+00 & 0.0000e+00 \\
\bottomrule
\end{tabular}

\end{table}
\begin{table}[h]
	\tiny
	\begin{tabular}{rrrrrrrrrrrrrrrr}
\toprule
       Y\textasciicircum 1 &        Y\textasciicircum 2 &        Y\textasciicircum 3 &        Y\textasciicircum 4 &        Y\textasciicircum 5 &         Y\textasciicircum 6 &        Y\textasciicircum 7 &         Y\textasciicircum 8 &        Y\textasciicircum 9 &        Y\textasciicircum 10 &       Y\textasciicircum 11 &        Y\textasciicircum 12 &       Y\textasciicircum 13 &       Y\textasciicircum 14 &       Y\textasciicircum 15 &       Y\textasciicircum 16 \\
\midrule
0.0000e+00 & 2.3967e-06 & 0.0000e+00 & 3.2821e-11 & 0.0000e+00 & -5.2308e-15 & 0.0000e+00 & -1.2575e-18 & 0.0000e+00 & -2.3505e-22 & 0.0000e+00 & -3.9908e-26 & 0.0000e+00 & 0.0000e+00 & 0.0000e+00 & 0.0000e+00 \\
\bottomrule
\end{tabular}

\end{table}

Biconic Zernike polynomial

\zernikeeq


\newpage
\section*{Lens 3}
\begin{table}[h]
	\tiny
	\begin{tabular}{lrrrrrrr}
\toprule
       SurfType &     Radius &  Thickness &  Semi-Diameter &       Conic &   X Radius &     X Conic &  Norm Radius \\
\midrule
Biconic Zernike & 3.1833e+02 & 1.4152e+01 &     8.2013e+01 & -1.6820e+01 & 3.7107e+02 & -7.3054e+00 &   1.0000e+02 \\
\bottomrule
\end{tabular}

\end{table}
\begin{table}[h]
	\tiny
	\begin{tabular}{rrrrrrrrrrrrrrrr}
\toprule
       X\textasciicircum 1 &         X\textasciicircum 2 &        X\textasciicircum 3 &        X\textasciicircum 4 &        X\textasciicircum 5 &        X\textasciicircum 6 &        X\textasciicircum 7 &         X\textasciicircum 8 &        X\textasciicircum 9 &        X\textasciicircum 10 &       X\textasciicircum 11 &        X\textasciicircum 12 &       X\textasciicircum 13 &       X\textasciicircum 14 &       X\textasciicircum 15 &       X\textasciicircum 16 \\
\midrule
0.0000e+00 & -9.8282e-07 & 0.0000e+00 & 1.9185e-10 & 0.0000e+00 & 7.6709e-15 & 0.0000e+00 & -3.5975e-18 & 0.0000e+00 & -1.5933e-21 & 0.0000e+00 & -4.9065e-25 & 0.0000e+00 & 0.0000e+00 & 0.0000e+00 & 0.0000e+00 \\
\bottomrule
\end{tabular}

\end{table}
\begin{table}[h]
	\tiny
	\begin{tabular}{rrrrrrrrrrrrrrrr}
\toprule
       Y\textasciicircum 1 &        Y\textasciicircum 2 &        Y\textasciicircum 3 &         Y\textasciicircum 4 &        Y\textasciicircum 5 &         Y\textasciicircum 6 &        Y\textasciicircum 7 &         Y\textasciicircum 8 &        Y\textasciicircum 9 &        Y\textasciicircum 10 &       Y\textasciicircum 11 &        Y\textasciicircum 12 &       Y\textasciicircum 13 &       Y\textasciicircum 14 &       Y\textasciicircum 15 &       Y\textasciicircum 16 \\
\midrule
0.0000e+00 & 2.7640e-06 & 0.0000e+00 & -3.3734e-10 & 0.0000e+00 & -6.3985e-14 & 0.0000e+00 & -9.5073e-18 & 0.0000e+00 & -1.1712e-21 & 0.0000e+00 & -5.4911e-26 & 0.0000e+00 & 0.0000e+00 & 0.0000e+00 & 0.0000e+00 \\
\bottomrule
\end{tabular}

\end{table}

Biconic Zernike polynomial

\zernikeeq


\end{document}